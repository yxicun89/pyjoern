\documentclass[12pt,a4paper]{article}
\usepackage[utf8]{inputenc}
\usepackage[japanese]{babel}
\usepackage{amsmath}
\usepackage{amsfonts}
\usepackage{amssymb}
\usepackage{graphicx}
\usepackage{listings}
\usepackage{xcolor}
\usepackage{hyperref}
\usepackage{geometry}
\usepackage{fancyhdr}
\usepackage{tocloft}

\geometry{margin=2.5cm}
\pagestyle{fancy}
\fancyhf{}
\fancyhead[L]{PyJoern完全ガイド}
\fancyhead[R]{\thepage}

% コードスタイル設定
\lstset{
    language=Python,
    basicstyle=\ttfamily\small,
    keywordstyle=\color{blue},
    commentstyle=\color{green},
    stringstyle=\color{red},
    numberstyle=\tiny\color{gray},
    numbers=left,
    frame=single,
    breaklines=true,
    showstringspaces=false,
    tabsize=4
}

\title{\textbf{PyJoern完全ガイド}\\
\large Pythonコード静的解析のための包括的ドキュメント}
\author{静的解析研究チーム}
\date{\today}

\begin{document}

\maketitle
\tableofcontents
\newpage

\section{はじめに}

PyJoernは、Pythonコードの静的解析を行うための強力なツールです。このドキュメントでは、PyJoernの基本的な使用方法から高度な分析手法まで、実際のプロジェクトで得られた知見をもとに詳しく解説します。

\subsection{PyJoernとは}

PyJoernは、Joern(静的解析プラットフォーム)のPython用バインディングです。コードプロパティグラフ(CPG: Code Property Graph)を生成し、制御フローグラフ(CFG)、抽象構文木(AST)、データ依存グラフ(DDG)などの解析を可能にします。

\subsection{主な機能}

\begin{itemize}
    \item 制御フローグラフ(CFG)の生成と解析
    \item 抽象構文木(AST)の解析
    \item データ依存グラフ(DDG)の生成
    \item 循環複雑度の計算
    \item 変数の読み書き分析
    \item パス分析
    \item 制御構造の検出
\end{itemize}

\section{環境設定}

\subsection{インストール}

PyJoernの環境設定は以下の手順で行います:

\begin{lstlisting}[language=bash, caption=仮想環境の作成とPyJoernのインストール]
# 仮想環境の作成
python -m venv pyjoern

# 仮想環境の有効化(Windows)
pyjoern\Scripts\activate

# 仮想環境の有効化(Linux/Mac)
source pyjoern/bin/activate

# PyJoernのインストール
pip install pyjoern

# 必要な依存関係
pip install networkx matplotlib
\end{lstlisting}

\subsection{動作確認}

\begin{lstlisting}[caption=基本的な動作確認]
from pyjoern import parse_source
import networkx as nx

# テストファイルの解析
functions = parse_source("test.py")
print(f"検出された関数: {list(functions.keys())}")
\end{lstlisting}

\section{基本的な使用方法}

\subsection{PyJoernの基本API}

PyJoernの主要なAPIを以下に示します:

\begin{lstlisting}[caption=PyJoernの基本API]
from pyjoern import parse_source, fast_cfgs_from_source

# 完全なパース解析
functions = parse_source("example.py")

# 高速CFG生成
cfgs = fast_cfgs_from_source("example.py")

# 各関数の情報取得
for func_name, func_obj in functions.items():
    print(f"関数名: {func_name}")
    print(f"開始行: {func_obj.start_line}")
    print(f"終了行: {func_obj.end_line}")

    # CFGの取得
    if func_obj.cfg:
        print(f"CFGノード数: {len(func_obj.cfg.nodes)}")
        print(f"CFGエッジ数: {len(func_obj.cfg.edges)}")

    # ASTの取得
    if func_obj.ast:
        print(f"ASTノード数: {len(func_obj.ast.nodes)}")
\end{lstlisting}

\subsection{サンプルコードの解析}

以下のサンプルコードを用いて、具体的な解析方法を説明します:

\begin{lstlisting}[caption=解析対象のサンプルコード(whiletest.py)]
def example(x):
    if x > 0:
        for i in range(x):
            print(i)
    else:
        while x < 0:
            x += 1
    return x

if __name__ == "__main__":
    print(example(5))
\end{lstlisting}

\section{制御フローグラフ(CFG)解析}

\subsection{CFGの基本構造}

制御フローグラフは、プログラムの実行経路を表現するグラフです。各ノードは基本ブロック(Basic Block)を表し、エッジは制御の流れを示します。

\begin{lstlisting}[caption=CFGノードの詳細分析]
def analyze_cfg(cfg):
    """CFGの詳細分析"""

    # エントリーポイントとエグジットポイントの特定
    entry_nodes = [n for n in cfg.nodes if n.is_entrypoint]
    exit_nodes = [n for n in cfg.nodes if n.is_exitpoint]

    print(f"エントリーポイント: {len(entry_nodes)}個")
    print(f"エグジットポイント: {len(exit_nodes)}個")

    # 各ノードの詳細分析
    for i, node in enumerate(cfg.nodes):
        print(f"\nノード {i+1} (Block:{node.addr}):")

        # ノードの属性
        print(f"  エントリーポイント: {node.is_entrypoint}")
        print(f"  エグジットポイント: {node.is_exitpoint}")
        print(f"  マージノード: {node.is_merged_node}")

        # ステートメントの解析
        if hasattr(node, 'statements') and node.statements:
            for j, stmt in enumerate(node.statements):
                stmt_type = type(stmt).__name__
                stmt_str = str(stmt)
                print(f"    Statement {j+1}: [{stmt_type}] {stmt_str}")
\end{lstlisting}

\subsection{CFGノードの種類}

PyJoernのCFGノードには以下の種類があります:

\begin{itemize}
    \item \textbf{Compare}: 比較演算(条件分岐)
    \item \textbf{Call}: 関数呼び出し
    \item \textbf{Assignment}: 代入操作
    \item \textbf{Return}: 関数の戻り値
    \item \textbf{Nop}: 特殊な操作(FUNCTION\_START、FUNCTION\_ENDなど)
    \item \textbf{UnsupportedStmt}: サポートされていないステートメント
\end{itemize}

\section{条件分岐とループの検出}

\subsection{条件分岐の検出}

条件分岐は主にCompare文として検出されます:

\begin{lstlisting}[caption=条件分岐の検出]
def detect_conditionals(cfg):
    """条件分岐の検出"""
    conditionals = []

    for node in cfg.nodes:
        if hasattr(node, 'statements') and node.statements:
            for stmt in node.statements:
                stmt_type = type(stmt).__name__
                stmt_str = str(stmt)

                # Compare文の検出
                if stmt_type == 'Compare':
                    conditionals.append({
                        'statement': stmt_str,
                        'node': node.addr,
                        'type': 'comparison'
                    })

    return conditionals
\end{lstlisting}

\subsection{ループの検出}

ループは複数の形式で検出できます:

\begin{lstlisting}[caption=ループの検出]
def detect_loops(cfg):
    """ループの検出"""
    loops = []

    for node in cfg.nodes:
        if hasattr(node, 'statements') and node.statements:
            for stmt in node.statements:
                stmt_type = type(stmt).__name__
                stmt_str = str(stmt)

                # for文(range呼び出し)の検出
                if stmt_type == 'Call' and 'range(' in stmt_str:
                    loops.append({
                        'type': 'for_loop',
                        'statement': stmt_str,
                        'node': node.addr
                    })

                # while文(iterator例外)の検出
                elif 'iteratorNonEmptyOrException' in stmt_str:
                    loops.append({
                        'type': 'while_loop',
                        'statement': stmt_str,
                        'node': node.addr
                    })

    return loops
\end{lstlisting}

\section{変数の読み書き分析}

\subsection{基本的な変数分析}

変数の読み書きパターンを正確に検出することは、静的解析において重要です:

\begin{lstlisting}[caption=変数の読み書き分析]
import re
from collections import defaultdict

def analyze_variables(cfg):
    """変数の読み書き分析"""
    variables = defaultdict(lambda: {'reads': 0, 'writes': 0})

    for node in cfg.nodes:
        if hasattr(node, 'statements') and node.statements:
            for stmt in node.statements:
                stmt_str = str(stmt)
                stmt_type = type(stmt).__name__

                # 通常の代入(書き込み)
                if stmt_type == 'Assignment':
                    match = re.match(r'(\w+)\s*=', stmt_str)
                    if match:
                        var_name = match.group(1)
                        variables[var_name]['writes'] += 1

                # 複合代入演算子(+= など)の検出
                elif (stmt_type == 'UnsupportedStmt' and
                      'assignmentPlus' in stmt_str):
                    match = re.search(r"'([a-zA-Z_]\w*)\s*\+=", stmt_str)
                    if match:
                        var_name = match.group(1)
                        variables[var_name]['reads'] += 1   # 読み取り
                        variables[var_name]['writes'] += 1  # 書き込み

                # 比較での変数使用(読み取り)
                elif stmt_type == 'Compare':
                    for var_name in re.findall(r'\b([a-zA-Z_]\w*)\b', stmt_str):
                        if var_name not in ['Compare', 'tmp0', 'tmp1']:
                            variables[var_name]['reads'] += 1

                # 関数呼び出しでの変数使用(読み取り)
                elif stmt_type == 'Call':
                    for var_name in re.findall(r'\b([a-zA-Z_]\w*)\b', stmt_str):
                        if var_name not in ['range', 'print', 'Call']:
                            variables[var_name]['reads'] += 1

    return dict(variables)
\end{lstlisting}

\subsection{複合代入演算子の処理}

複合代入演算子(+=、-=、*=など)は読み書き両方の操作を行うため、特別な処理が必要です:

\begin{lstlisting}[caption=複合代入演算子の詳細検出]
def detect_compound_assignments(cfg):
    """複合代入演算子の検出"""
    compound_ops = []

    for node in cfg.nodes:
        if hasattr(node, 'statements') and node.statements:
            for stmt in node.statements:
                stmt_str = str(stmt)
                stmt_type = type(stmt).__name__

                if stmt_type == 'UnsupportedStmt':
                    # += の検出
                    if 'assignmentPlus' in stmt_str:
                        match = re.search(r"'([a-zA-Z_]\w*)\s*\+=", stmt_str)
                        if match:
                            compound_ops.append({
                                'variable': match.group(1),
                                'operator': '+=',
                                'node': node.addr
                            })

                    # -= の検出
                    elif 'assignmentMinus' in stmt_str:
                        match = re.search(r"'([a-zA-Z_]\w*)\s*-=", stmt_str)
                        if match:
                            compound_ops.append({
                                'variable': match.group(1),
                                'operator': '-=',
                                'node': node.addr
                            })

    return compound_ops
\end{lstlisting}

\section{循環複雑度の計算}

\subsection{McCabe循環複雑度}

循環複雑度は、プログラムの複雑さを測る重要な指標です。正確な計算式は以下の通りです:

\begin{equation}
M = E - N + 2P
\end{equation}

ここで:
\begin{itemize}
    \item M: McCabe循環複雑度
    \item E: グラフのエッジ数
    \item N: グラフのノード数
    \item P: 連結成分の個数
\end{itemize}

\begin{lstlisting}[caption=正確なMcCabe複雑度の計算]
def calculate_mccabe_complexity(cfg):
    """McCabe循環複雑度の計算"""

    # グラフの基本情報
    E = len(cfg.edges)  # エッジ数
    N = len(cfg.nodes)  # ノード数

    # 連結成分数の計算
    try:
        undirected_cfg = cfg.to_undirected()
        P = nx.number_connected_components(undirected_cfg)
    except:
        P = 1  # エラーの場合は1とする

    # McCabe複雑度: M = E - N + 2P
    mccabe_complexity = E - N + 2 * P

    # 簡易計算(参考値)
    conditional_count = count_conditionals(cfg)
    loop_count = count_loops(cfg)
    simple_complexity = conditional_count + loop_count + 1

    return {
        'mccabe_complexity': mccabe_complexity,
        'simple_complexity': simple_complexity,
        'edges': E,
        'nodes': N,
        'components': P,
        'formula': f"M = {E} - {N} + 2×{P} = {mccabe_complexity}"
    }
\end{lstlisting}

\subsection{複雑度の解釈}

循環複雑度の一般的な解釈は以下の通りです:

\begin{itemize}
    \item 1-10: 低い複雑度(良好)
    \item 11-20: 中程度の複雑度(注意)
    \item 21-50: 高い複雑度(リファクタリング推奨)
    \item 51以上: 非常に高い複雑度(緊急対応必要)
\end{itemize}

\section{パス分析}

\subsection{静的パス分析}

静的解析では、プログラムの構造的な実行経路を分析します:

\begin{lstlisting}[caption=パス分析の実装]
def analyze_paths(cfg):
    """CFGのパス分析"""

    # エントリーポイントとエグジットポイントの特定
    entry_nodes = [n for n in cfg.nodes if n.is_entrypoint]
    exit_nodes = [n for n in cfg.nodes if n.is_exitpoint]

    if not entry_nodes or not exit_nodes:
        return {'total_paths': 0, 'path_details': []}

    total_paths = 0
    all_paths = []

    # すべての単純パスを計算
    for entry in entry_nodes:
        for exit in exit_nodes:
            try:
                simple_paths = list(nx.all_simple_paths(cfg, entry, exit))
                total_paths += len(simple_paths)
                all_paths.extend(simple_paths)
            except nx.NetworkXNoPath:
                pass

    return {
        'total_paths': total_paths,
        'path_details': all_paths[:5]  # 最初の5つのパスを表示
    }
\end{lstlisting}

\subsection{パス数の推定}

条件分岐とループに基づく実行可能パス数の推定:

\begin{lstlisting}[caption=実行可能パス数の推定]
def estimate_executable_paths(cfg):
    """実行可能パス数の推定"""

    conditional_count = count_conditionals(cfg)
    loop_count = count_loops(cfg)

    # 基本推定: 2^(条件分岐数)
    estimated_paths = 2 ** conditional_count

    # ループがある場合の調整
    if loop_count > 0:
        estimated_paths *= 3  # 0回、1回、複数回実行の可能性

    return {
        'estimated_paths': estimated_paths,
        'conditional_branches': conditional_count,
        'loops': loop_count
    }
\end{lstlisting}

\section{実践的な使用例}

\subsection{包括的分析スクリプト}

以下は、PyJoernを使用した包括的な分析を行うスクリプトの例です:

\begin{lstlisting}[caption=包括的分析の実装]
def comprehensive_analysis(file_path):
    """包括的な静的解析"""

    functions = parse_source(file_path)
    results = {}

    for func_name, func_obj in functions.items():
        if func_obj.cfg and isinstance(func_obj.cfg, nx.DiGraph):

            # 基本メトリクス
            metrics = {
                'connected_components': analyze_connected_components(func_obj.cfg),
                'loop_statements': count_loops(func_obj.cfg),
                'conditional_statements': count_conditionals(func_obj.cfg),
                'cycles': analyze_cycles(func_obj.cfg),
                'paths': analyze_paths(func_obj.cfg),
                'cyclomatic_complexity': calculate_mccabe_complexity(func_obj.cfg),
                'variables': analyze_variables(func_obj.cfg)
            }

            results[func_name] = metrics

    return results
\end{lstlisting}

\subsection{結果の可視化}

分析結果を表形式で出力する例:

\begin{lstlisting}[caption=結果の表示]
def display_results(results):
    """分析結果の表示"""

    for func_name, metrics in results.items():
        print(f"\n関数: {func_name}")
        print("-" * 40)

        print(f"Connected Components     : {metrics['connected_components']}")
        print(f"Loop Statements          : {metrics['loop_statements']}")
        print(f"Conditional Statements   : {metrics['conditional_statements']}")
        print(f"Cycles                   : {metrics['cycles']}")
        print(f"Paths                    : {metrics['paths']['total_paths']}")
        print(f"Cyclomatic Complexity    : {metrics['cyclomatic_complexity']['mccabe_complexity']}")

        # 変数分析結果
        var_metrics = metrics['variables']
        print(f"Variable Count           : {len(var_metrics)}")

        total_reads = sum(v['reads'] for v in var_metrics.values())
        total_writes = sum(v['writes'] for v in var_metrics.values())
        print(f"Total Reads              : {total_reads}")
        print(f"Total Writes             : {total_writes}")
\end{lstlisting}

\section{トラブルシューティング}

\subsection{よくある問題と解決方法}

\subsubsection{NetworkXのインポートエラー}

\begin{lstlisting}[language=bash]
# NetworkXが見つからない場合
pip install networkx
\end{lstlisting}

\subsubsection{PyJoernの解析エラー}

\begin{lstlisting}[caption=エラーハンドリングの例]
try:
    functions = parse_source(file_path)
except Exception as e:
    print(f"解析エラー: {e}")
    # フォールバック処理
    functions = {}
\end{lstlisting}

\subsubsection{複合代入演算子が検出されない}

複合代入演算子(+=、-=など)は `UnsupportedStmt` として扱われるため、特別な正規表現での検出が必要です。

\subsection{パフォーマンスの最適化}

大きなファイルを解析する際のパフォーマンス向上のヒント:

\begin{itemize}
    \item `fast_cfgs_from_source()` を使用する
    \item 分析する関数を限定する
    \item メモリ使用量に注意する
\end{itemize}

\section{ベストプラクティス}

\subsection{効率的な分析のための推奨事項}

\begin{enumerate}
    \item \textbf{段階的な分析}: 基本的なメトリクスから始めて、必要に応じて詳細な分析を行う
    \item \textbf{エラーハンドリング}: PyJoernの解析が失敗する可能性を考慮する
    \item \textbf{結果の検証}: 複数の方法で同じメトリクスを計算し、結果を検証する
    \item \textbf{ドキュメント化}: 分析結果とその解釈を適切に文書化する
\end{enumerate}

\subsection{分析精度向上のためのテクニック}

\begin{itemize}
    \item 複合代入演算子の適切な処理
    \item 一時変数の除外
    \item ライブラリ関数の識別
    \item ネストした制御構造の正確な検出
\end{itemize}

\section{応用例}

\subsection{コード品質評価}

PyJoernを使用してコード品質を評価する例:

\begin{lstlisting}[caption=コード品質評価]
def evaluate_code_quality(file_path):
    """コード品質の評価"""

    results = comprehensive_analysis(file_path)
    quality_score = 0

    for func_name, metrics in results.items():
        complexity = metrics['cyclomatic_complexity']['mccabe_complexity']

        # 複雑度に基づくスコア
        if complexity <= 10:
            quality_score += 100
        elif complexity <= 20:
            quality_score += 70
        else:
            quality_score += 30

    return quality_score / len(results) if results else 0
\end{lstlisting}

\subsection{リファクタリング候補の特定}

複雑度の高い関数を特定し、リファクタリングの優先順位を決定:

\begin{lstlisting}[caption=リファクタリング候補の特定]
def identify_refactoring_candidates(file_path):
    """リファクタリング候補の特定"""

    results = comprehensive_analysis(file_path)
    candidates = []

    for func_name, metrics in results.items():
        complexity = metrics['cyclomatic_complexity']['mccabe_complexity']
        paths = metrics['paths']['total_paths']

        if complexity > 10 or paths > 5:
            candidates.append({
                'function': func_name,
                'complexity': complexity,
                'paths': paths,
                'priority': complexity + paths
            })

    # 優先順位でソート
    candidates.sort(key=lambda x: x['priority'], reverse=True)
    return candidates
\end{lstlisting}

\section{高度な分析技法}

\subsection{関数間の依存関係分析}

PyJoernでは、関数間の呼び出し関係も分析できます:

\begin{lstlisting}[caption=関数間依存関係の分析]
def analyze_function_dependencies(functions):
    """関数間の依存関係を分析"""
    dependencies = defaultdict(set)

    for func_name, func_obj in functions.items():
        if func_obj.cfg:
            for node in func_obj.cfg.nodes:
                if hasattr(node, 'statements') and node.statements:
                    for stmt in node.statements:
                        if type(stmt).__name__ == 'Call':
                            stmt_str = str(stmt)
                            # 他の関数への呼び出しを検出
                            for other_func in functions.keys():
                                if other_func != func_name and other_func in stmt_str:
                                    dependencies[func_name].add(other_func)

    return dict(dependencies)
\end{lstlisting}

\subsection{データフロー分析}

変数のライフサイクルを追跡:

\begin{lstlisting}[caption=変数のライフサイクル分析]
def analyze_variable_lifecycle(cfg):
    """変数のライフサイクル分析"""
    lifecycle = defaultdict(list)

    for node in cfg.nodes:
        if hasattr(node, 'statements') and node.statements:
            for stmt in node.statements:
                stmt_str = str(stmt)
                stmt_type = type(stmt).__name__

                # 変数の定義
                if stmt_type == 'Assignment':
                    match = re.match(r'(\w+)\s*=', stmt_str)
                    if match:
                        var_name = match.group(1)
                        lifecycle[var_name].append({
                            'action': 'definition',
                            'node': node.addr,
                            'statement': stmt_str
                        })

                # 変数の使用
                elif stmt_type == 'Compare':
                    for var_name in re.findall(r'\b([a-zA-Z_]\w*)\b', stmt_str):
                        if var_name not in ['Compare', 'tmp0', 'tmp1']:
                            lifecycle[var_name].append({
                                'action': 'usage',
                                'node': node.addr,
                                'statement': stmt_str
                            })

    return dict(lifecycle)
\end{lstlisting}

\subsection{コードメトリクス集計}

複数のメトリクスを統合して評価:

\begin{lstlisting}[caption=総合的なコードメトリクス]
def calculate_comprehensive_metrics(file_path):
    """総合的なコードメトリクス計算"""
    functions = parse_source(file_path)

    overall_metrics = {
        'total_functions': len(functions),
        'total_complexity': 0,
        'average_complexity': 0,
        'max_complexity': 0,
        'total_paths': 0,
        'total_variables': 0,
        'function_details': {}
    }

    for func_name, func_obj in functions.items():
        if func_obj.cfg:
            complexity = calculate_mccabe_complexity(func_obj.cfg)
            paths = analyze_paths(func_obj.cfg)
            variables = analyze_variables(func_obj.cfg)

            func_complexity = complexity['mccabe_complexity']
            func_paths = paths['total_paths']
            func_variables = len(variables)

            overall_metrics['total_complexity'] += func_complexity
            overall_metrics['total_paths'] += func_paths
            overall_metrics['total_variables'] += func_variables
            overall_metrics['max_complexity'] = max(
                overall_metrics['max_complexity'], func_complexity
            )

            overall_metrics['function_details'][func_name] = {
                'complexity': func_complexity,
                'paths': func_paths,
                'variables': func_variables
            }

    # 平均値の計算
    if overall_metrics['total_functions'] > 0:
        overall_metrics['average_complexity'] = (
            overall_metrics['total_complexity'] / overall_metrics['total_functions']
        )

    return overall_metrics
\end{lstlisting}

\section{継続的インテグレーション(CI)での活用}

\subsection{品質ゲートの実装}

\begin{lstlisting}[caption=CI用品質チェック]
def quality_gate_check(file_path, thresholds=None):
    """CI用の品質ゲートチェック"""

    if thresholds is None:
        thresholds = {
            'max_complexity': 10,
            'max_paths': 5,
            'max_variables': 15
        }

    results = comprehensive_analysis(file_path)
    failures = []

    for func_name, metrics in results.items():
        complexity = metrics['cyclomatic_complexity']['mccabe_complexity']
        paths = metrics['paths']['total_paths']
        variables = len(metrics['variables'])

        if complexity > thresholds['max_complexity']:
            failures.append({
                'function': func_name,
                'metric': 'complexity',
                'value': complexity,
                'threshold': thresholds['max_complexity']
            })

        if paths > thresholds['max_paths']:
            failures.append({
                'function': func_name,
                'metric': 'paths',
                'value': paths,
                'threshold': thresholds['max_paths']
            })

        if variables > thresholds['max_variables']:
            failures.append({
                'function': func_name,
                'metric': 'variables',
                'value': variables,
                'threshold': thresholds['max_variables']
            })

    return {
        'passed': len(failures) == 0,
        'failures': failures
    }
\end{lstlisting}

\subsection{レポート生成}

\begin{lstlisting}[caption=HTML品質レポート生成]
def generate_html_report(file_path, output_file="quality_report.html"):
    """HTML品質レポートの生成"""

    metrics = calculate_comprehensive_metrics(file_path)

    html_content = f"""
    <!DOCTYPE html>
    <html>
    <head>
        <title>コード品質レポート</title>
        <style>
            body {{ font-family: Arial, sans-serif; margin: 20px; }}
            table {{ border-collapse: collapse; width: 100%; }}
            th, td {{ border: 1px solid #ddd; padding: 8px; text-align: left; }}
            th {{ background-color: #f2f2f2; }}
            .high-complexity {{ background-color: #ffcccc; }}
            .medium-complexity {{ background-color: #ffffcc; }}
            .low-complexity {{ background-color: #ccffcc; }}
        </style>
    </head>
    <body>
        <h1>コード品質レポート</h1>
        <h2>概要</h2>
        <p>総関数数: {metrics['total_functions']}</p>
        <p>平均複雑度: {metrics['average_complexity']:.2f}</p>
        <p>最大複雑度: {metrics['max_complexity']}</p>

        <h2>関数別詳細</h2>
        <table>
            <tr>
                <th>関数名</th>
                <th>複雑度</th>
                <th>実行パス数</th>
                <th>変数数</th>
                <th>評価</th>
            </tr>
    """

    for func_name, details in metrics['function_details'].items():
        complexity = details['complexity']

        if complexity <= 10:
            css_class = "low-complexity"
            evaluation = "良好"
        elif complexity <= 20:
            css_class = "medium-complexity"
            evaluation = "注意"
        else:
            css_class = "high-complexity"
            evaluation = "改善必要"

        html_content += f"""
            <tr class="{css_class}">
                <td>{func_name}</td>
                <td>{complexity}</td>
                <td>{details['paths']}</td>
                <td>{details['variables']}</td>
                <td>{evaluation}</td>
            </tr>
        """

    html_content += """
        </table>
    </body>
    </html>
    """

    with open(output_file, 'w', encoding='utf-8') as f:
        f.write(html_content)

    return output_file
\end{lstlisting}

\section{まとめ}

PyJoernは、Pythonコードの静的解析において非常に有用なツールです。このドキュメントで紹介した手法を用いることで、以下のような分析が可能になります:

\begin{itemize}
    \item 正確な循環複雑度の計算
    \item 詳細な制御フロー分析
    \item 変数の読み書きパターンの解析
    \item 実行パスの特定と評価
    \item コード品質の定量的評価
    \item 継続的インテグレーションでの自動品質チェック
\end{itemize}

特に重要なポイントは以下の通りです:

\begin{enumerate}
    \item McCabe複雑度は M = E - N + 2P の正確な式で計算する
    \item while文の条件も条件分岐としてカウントする
    \item 複合代入演算子(+=など)は読み書き両方として扱う
    \item 静的解析では構造的なパス数を計算する
    \item エラーハンドリングを適切に行う
    \item 継続的な品質監視体制を構築する
\end{enumerate}

これらの知見を活用することで、PyJoernを使った効果的な静的解析が可能になります。

\section{参考資料}

\begin{itemize}
    \item PyJoern公式ドキュメント: \url{https://github.com/fabsx00/pyjoern}
    \item Joernプロジェクト: \url{https://joern.io/}
    \item NetworkX文書: \url{https://networkx.org/}
    \item McCabe循環複雑度: McCabe, T.J. (1976). "A Complexity Measure"
\end{itemize}

\appendix

\section{サンプルコード集}

\subsection{完全な分析スクリプト}

\begin{lstlisting}[caption=完全な分析スクリプトの例, breaklines=true]
#!/usr/bin/env python3
# -*- coding: utf-8 -*-

from pyjoern import parse_source
import networkx as nx
from collections import defaultdict
import re

def main():
    """メイン実行関数"""
    file_path = "whiletest.py"

    print("="*80)
    print("PyJoern包括的分析")
    print("="*80)

    # 分析実行
    results = comprehensive_analysis(file_path)

    # 結果表示
    display_results(results)

    # 品質評価
    quality = evaluate_code_quality(file_path)
    print(f"\nコード品質スコア: {quality:.1f}/100")

if __name__ == "__main__":
    main()
\end{lstlisting}

\end{document}
