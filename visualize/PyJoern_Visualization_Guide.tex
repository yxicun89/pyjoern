\documentclass[a4paper,12pt]{article}
\usepackage[utf8]{inputenc}
\usepackage[japanese]{babel}
\usepackage{geometry}
\usepackage{xcolor}
\usepackage{listings}
\usepackage{fancyhdr}
\usepackage{titlesec}
\usepackage{booktabs}
\usepackage{graphicx}
\usepackage{hyperref}

\geometry{margin=2.5cm}
\setlength{\parindent}{0pt}
\setlength{\parskip}{6pt}

% コードリスティングの設定
\lstset{
    language=Python,
    basicstyle=\footnotesize\ttfamily,
    keywordstyle=\color{blue}\bfseries,
    commentstyle=\color{green!60!black},
    stringstyle=\color{red},
    showstringspaces=false,
    breaklines=true,
    frame=single,
    backgroundcolor=\color{gray!10},
    numbers=left,
    numberstyle=\tiny\color{gray},
    captionpos=b
}

% ヘッダー・フッターの設定
\pagestyle{fancy}
\fancyhf{}
\fancyhead[L]{PyJoern 視覚化プログラム解説}
\fancyhead[R]{\today}
\fancyfoot[C]{\thepage}

% セクションタイトルの設定
\titleformat{\section}{\Large\bfseries\color{blue!70!black}}{\thesection}{1em}{}
\titleformat{\subsection}{\large\bfseries\color{blue!50!black}}{\thesubsection}{1em}{}
\titleformat{\subsubsection}{\normalsize\bfseries\color{blue!30!black}}{\thesubsubsection}{1em}{}

\title{\Huge\bfseries PyJoern 視覚化プログラム解説}
\author{静的解析・視覚化ツール完全ガイド}
\date{\today}

\begin{document}

\maketitle
\tableofcontents
\newpage

\section{概要}

\texttt{visualize\_graphs\_fixed.py} は、Pythonソースコードを静的解析してCFG(制御フローグラフ)、AST(抽象構文木)、DDG(データ依存グラフ)を視覚化するプログラムです。PyJoernライブラリを使用して、複雑なプログラムの制御フローを直感的に理解できる強力な静的解析・視覚化ツールを提供します。

\section{プログラム全体構成}

\subsection{メイン機能}

\begin{lstlisting}[caption=メイン関数]
def analyze_and_visualize_file(source_file, output_dir="graph_images")
\end{lstlisting}

このプログラムは以下の3つの主要機能を提供します:

\begin{enumerate}
    \item \textbf{parse\_source解析}: 詳細な静的解析でCFG、AST、DDGを生成
    \item \textbf{fast\_cfg解析}: 高速CFG解析で関数レベルのCFGのみを生成
    \item \textbf{グラフ比較}: 3つのグラフを横並びで比較表示
\end{enumerate}

\section{視覚化の核心機能}

\subsection{ノードラベル作成 (\texttt{create\_node\_labels})}

\begin{lstlisting}[caption=ノードラベル作成関数]
def create_node_labels(graph, graph_type="CFG"):
\end{lstlisting}

\subsubsection{CFGノードの特徴}

\begin{itemize}
    \item \textbf{Block番号}: \texttt{Block \{node.addr\}} でブロックを識別
    \item \textbf{ステートメント分析}: 汎用的なパターン検出
    \begin{itemize}
        \item \texttt{LOOP\_ITERATION}: ループの反復チェック
        \item \texttt{CONDITION}: 条件比較(\texttt{Compare:}で検出)
        \item \texttt{FUNCTION\_CALL}: 関数呼び出し(\texttt{Call:}で検出)
        \item \texttt{ASSIGNMENT}: 代入文(\texttt{Assignment:}で検出)
        \item \texttt{FUNCTION\_START/END}: 関数の開始/終了
    \end{itemize}
\end{itemize}

\subsubsection{特別処理}

\begin{lstlisting}[caption=特別処理の例]
# FUNCTION_START + 条件判定の組み合わせノード
if has_function_start and has_condition:
    base_label = "[START] FUNCTION ENTRY\n" + "="*16
\end{lstlisting}

\subsection{エッジの色分けとスタイル (\texttt{get\_edge\_colors\_and\_styles})}

\begin{lstlisting}[caption=エッジ色分け関数]
def get_edge_colors_and_styles(graph, graph_type="CFG"):
\end{lstlisting}

\subsubsection{エッジタイプの識別}

\begin{itemize}
    \item \textcolor{red}{\textbf{ループバックエッジ}}: \texttt{\#FF6B6B}(赤色、破線)
    \begin{itemize}
        \item \texttt{source.addr > target.addr} でループを検出
        \item 関数終了ノードからのエッジは除外
    \end{itemize}
    \item \textcolor{blue}{\textbf{条件分岐エッジ}}: \texttt{\#4169E1}(青色、実線)
    \begin{itemize}
        \item \texttt{Compare:}文の存在で判定
    \end{itemize}
    \item \textbf{通常エッジ}: \texttt{\#333333}(ダークグレー、実線)
\end{itemize}

\subsection{ノードの色分け (\texttt{get\_node\_colors})}

\begin{lstlisting}[caption=ノード色分け関数]
def get_node_colors(graph, graph_type="CFG"):
\end{lstlisting}

\subsubsection{CFGノードの色コード}

\begin{table}[h]
\centering
\begin{tabular}{@{}llll@{}}
\toprule
\textbf{色} & \textbf{HEXコード} & \textbf{意味} & \textbf{説明} \\
\midrule
\textcolor{red}{●} & \texttt{\#FF6B6B} & \textbf{警告} & FUNCTION\_START + 条件判定の組み合わせ \\
\textcolor{green}{●} & \texttt{\#90EE90} & \textbf{エントリーポイント} & 関数開始 \\
\textcolor{pink}{●} & \texttt{\#FFB6C1} & \textbf{エグジットポイント} & 関数終了 \\
\textcolor{yellow}{●} & \texttt{\#FFD700} & \textbf{マージノード} & 制御フローの合流点 \\
\textcolor{lightblue}{●} & \texttt{\#87CEFA} & \textbf{条件判定ノード} & if/while文 \\
\textcolor{lightgreen}{●} & \texttt{\#98FB98} & \textbf{関数呼び出し} & Call文 \\
\textcolor{skyblue}{●} & \texttt{\#87CEEB} & \textbf{通常ノード} & その他のステートメント \\
\bottomrule
\end{tabular}
\caption{CFGノードの色分け一覧}
\label{tab:node_colors}
\end{table}

\section{グラフ描画機能}

\subsection{単一グラフ視覚化 (\texttt{visualize\_graph})}

\begin{lstlisting}[caption=単一グラフ視覚化関数]
def visualize_graph(graph, title, graph_type="CFG", save_path=None):
\end{lstlisting}

\subsubsection{レイアウト選択}

\begin{itemize}
    \item \textbf{小規模グラフ} (≤10ノード): Spring layout
    \item \textbf{CFG}: Graphviz dot layout(階層的)
    \item \textbf{AST/DDG}: Spring layout
\end{itemize}

\subsubsection{描画順序}

\begin{enumerate}
    \item エッジ描画(カーブ付き)
    \item ノード描画(色分け)
    \item ラベル描画
    \item エッジラベル描画("Loop"など)
    \item 凡例追加
\end{enumerate}

\subsection{比較グラフ (\texttt{compare\_graphs\_side\_by\_side})}

\begin{lstlisting}[caption=比較グラフ関数]
def compare_graphs_side_by_side(cfg, ast, ddg, func_name, save_dir=None):
\end{lstlisting}

CFG、AST、DDGを横並びで比較表示し、各グラフの特徴を一目で確認可能。

\section{高度な描画機能}

\subsection{カーブ付きエッジ描画 (\texttt{draw\_edges\_with\_curves})}

\begin{lstlisting}[caption=カーブ付きエッジ描画関数]
def draw_edges_with_curves(graph, pos, edge_colors, edge_styles):
\end{lstlisting}

\begin{itemize}
    \item \textbf{重複エッジ対応}: 同じノード間の複数エッジを異なるカーブで描画
    \item \textbf{カーブ角度}: 0.1、0.3の異なる弧度で視覚的に区別
\end{itemize}

\subsection{エッジラベル (\texttt{get\_edge\_labels})}

\begin{itemize}
    \item \textbf{"Loop"ラベル}: 実際のループバックエッジのみ
    \item \textbf{関数終了除外}: FUNCTION\_ENDノードからのエッジは除外
    \item \textbf{True/False除去}: CFGでは条件分岐ラベルは色分けのみ
\end{itemize}

\section{汎用性の工夫}

\subsection{パターンベース検出}

特定の変数名(\texttt{x > 0}、\texttt{i \% 2}など)にハードコーディングせず、汎用的なパターンで検出:

\begin{lstlisting}[caption=汎用的パターン検出]
# 条件比較の判定(汎用的)
elif 'Compare:' in stmt_str:
    primary_stmt_type = "CONDITION"

# ループ構造の判定
if 'iteratorNonEmptyOrException' in stmt_str:
    primary_stmt_type = "LOOP_ITERATION"
\end{lstlisting}

\subsection{HTMLエンティティ対応}

\begin{lstlisting}[caption=HTMLエンティティ対応]
# フィルタリング条件
if (not name.startswith('<operator>') and
    not name.startswith('&lt;operator&gt;') and  # HTMLエンコード対応
    not name == '<module>' and
    not name == '&lt;module&gt;'):
\end{lstlisting}

\section{実行フロー}

\begin{enumerate}
    \item \textbf{ソースファイル読み込み}
    \item \textbf{parse\_source解析} → CFG、AST、DDG生成
    \item \textbf{個別グラフ視覚化} → PNG保存
    \item \textbf{比較グラフ生成} → 3つのグラフを横並び表示
    \item \textbf{fast\_cfg解析} → 関数レベルCFGのみ抽出・視覚化
\end{enumerate}

\section{視覚的特徴}

\begin{itemize}
    \item \textbf{色分けによる識別}: エントリー/エグジット/条件/ループを色で区別
    \item \textbf{矢印スタイル}: 制御フローの方向を明確に表示
    \item \textbf{ラベル情報}: ブロック番号、ステートメント要約、行番号
    \item \textbf{凡例付き}: 各色とエッジタイプの意味を説明
\end{itemize}

\section{使用例}

\begin{lstlisting}[caption=基本的な使用例]
# whiletest.pyを解析
analyze_and_visualize_file("whiletest.py")
\end{lstlisting}

このプログラムは、複雑なプログラムの制御フローを直感的に理解できる強力な静的解析・視覚化ツールです。

\section{技術的詳細}

\subsection{依存ライブラリ}

\begin{itemize}
    \item \texttt{pyjoern}: 静的解析エンジン
    \item \texttt{networkx}: グラフ操作
    \item \texttt{matplotlib}: 視覚化
    \item \texttt{numpy}: 数値計算
\end{itemize}

\subsection{出力形式}

\begin{itemize}
    \item \textbf{個別グラフ}: PNG形式で保存
    \item \textbf{比較グラフ}: 3つのグラフを横並びで表示・保存
    \item \textbf{デバッグ情報}: コンソール出力で解析状況を確認
\end{itemize}

\subsection{カスタマイズ可能要素}

\begin{itemize}
    \item \textbf{色設定}: ノード・エッジの色を変更可能
    \item \textbf{レイアウト}: グラフレイアウトアルゴリズムを選択可能
    \item \textbf{ラベル内容}: ノードラベルの表示内容を調整可能
    \item \textbf{保存形式}: 画像形式・解像度を変更可能
\end{itemize}

\section{まとめ}

PyJoern視覚化プログラムは、静的解析による制御フローの理解を大幅に向上させる包括的なツールです。汎用的な設計により、様々なPythonプログラムに対応でき、視覚的に分かりやすいグラフ表示により、コードの構造把握が容易になります。

\end{document}
