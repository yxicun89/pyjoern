```tex
\documentclass[12pt,a4paper]{article}
\usepackage[utf8]{inputenc}
\usepackage[japanese]{babel}
\usepackage{amsmath}
\usepackage{amsfonts}
\usepackage{amssymb}
\usepackage{graphicx}
\usepackage{listings}
\usepackage{xcolor}
\usepackage{hyperref}
\usepackage{geometry}
\usepackage{fancyhdr}
\usepackage{tocloft}
\usepackage{array}
\usepackage{longtable}

\geometry{margin=2.5cm}
\pagestyle{fancy}
\fancyhf{}
\fancyhead[L]{PyJoern基本ガイド}
\fancyhead[R]{\thepage}

% コードスタイル設定
\lstdefinestyle{pythonstyle}{
    language=Python,
    basicstyle=\ttfamily\small,
    keywordstyle=\color{blue},
    commentstyle=\color{green!50!black},
    stringstyle=\color{red},
    numberstyle=\tiny\color{gray},
    breaklines=true,
    showstringspaces=false,
    frame=single,
    tabsize=4
}
\lstset{style=pythonstyle}

\title{\textbf{PyJoern基本ガイド}\\
\large Pythonコード静的解析ツールの基本的な使い方}
\author{静的解析研究チーム}
\date{\today}

\begin{document}

\maketitle
\tableofcontents
\newpage

\section{はじめに}

PyJoernは、Pythonコードの静的解析を行うためのツールです。Joern(静的解析プラットフォーム)のPython用バインディングとして、コードの構造を解析し、制御フローグラフ(CFG)や抽象構文木(AST)を生成します。

\subsection{主な機能}
\begin{itemize}
    \item 制御フローグラフ(CFG)の生成
    \item 抽象構文木(AST)の生成
    \item 関数の基本情報取得
    \item コードの構造解析
\end{itemize}

\section{環境設定}

\subsection{インストール}

\begin{lstlisting}[language=bash]
# 仮想環境の作成
python -m venv pyjoern

# 仮想環境の有効化(Windows)
pyjoern\Scripts\activate

# 仮想環境の有効化(Linux/Mac)
source pyjoern/bin/activate

# PyJoernのインストール
pip install pyjoern networkx
\end{lstlisting}

\subsection{動作確認}

\begin{lstlisting}
from pyjoern import parse_source
print("PyJoern installed successfully!")
\end{lstlisting}

\section{基本的なAPI}

PyJoernには主に2つのAPIがあります:

\subsection{parse\_source() - 完全な解析}
\begin{lstlisting}
from pyjoern import parse_source

functions = parse_source("example.py")
\end{lstlisting}

\subsection{fast\_cfgs\_from\_source() - 高速CFG生成}
\begin{lstlisting}
from pyjoern import fast_cfgs_from_source

cfgs = fast_cfgs_from_source("example.py")
\end{lstlisting}

\section{parse\_source vs fast\_cfgs\_from\_source}

\begin{table}[h!]
\centering
\begin{tabular}{|l|l|l|}
\hline
\textbf{項目} & \textbf{parse\_source} & \textbf{fast\_cfgs\_from\_source} \\
\hline
速度 & 遅い & 高速 \\
\hline
詳細度 & 高い(CFG + AST) & 中程度(CFGのみ) \\
\hline
メモリ使用量 & 多い & 少ない \\
\hline
用途 & 詳細分析 & 大量ファイル処理 \\
\hline
\end{tabular}
\caption{APIの比較}
\end{table}

\subsection{parse\_source()の特徴}

\begin{lstlisting}
functions = parse_source("example.py")
for func_name, func_obj in functions.items():
    print(f"関数名: {func_name}")
    print(f"CFG: {func_obj.cfg}")      # 制御フローグラフ
    print(f"AST: {func_obj.ast}")      # 抽象構文木
    print(f"開始行: {func_obj.start_line}")
    print(f"終了行: {func_obj.end_line}")
\end{lstlisting}

\textbf{取得できる情報:}
\begin{itemize}
    \item 関数の詳細情報(開始行、終了行)
    \item 高品質なCFG
    \item 完全なAST
    \item ノードの詳細な属性
\end{itemize}

\subsection{fast\_cfgs\_from\_source()の特徴}

\begin{lstlisting}
cfgs = fast_cfgs_from_source("example.py")
for cfg_name, cfg in cfgs.items():
    print(f"CFG名: {cfg_name}")
    print(f"ノード数: {len(cfg.nodes)}")
    print(f"エッジ数: {len(cfg.edges)}")
\end{lstlisting}

\textbf{取得できる情報:}
\begin{itemize}
    \item CFGのみ(ASTなし)
    \item 基本的なノード情報
    \item 高速な処理
\end{itemize}

\section{取得できるデータ}

\subsection{関数オブジェクト(parse\_source)}

\begin{lstlisting}
functions = parse_source("example.py")
func_obj = functions["example"]

# 基本情報
print(func_obj.start_line)  # 関数の開始行
print(func_obj.end_line)    # 関数の終了行

# グラフ情報
cfg = func_obj.cfg          # 制御フローグラフ
ast = func_obj.ast          # 抽象構文木
\end{lstlisting}

\subsection{CFGノードの構造}

\begin{lstlisting}
for node in cfg.nodes:
    print(f"ノード型: {type(node)}")
    print(f"アドレス: {node.addr}")
    print(f"インデックス: {node.idx}")

    # ノードの属性
    print(f"エントリーポイント: {node.is_entrypoint}")
    print(f"エグジットポイント: {node.is_exitpoint}")
    print(f"マージノード: {node.is_merged_node}")

    # ステートメント(実行文)
    if hasattr(node, 'statements'):
        for stmt in node.statements:
            print(f"ステートメント型: {type(stmt).__name__}")
            print(f"ステートメント内容: {stmt}")
\end{lstlisting}

\subsection{ステートメントの種類}

\begin{longtable}{|l|p{4cm}|p{4cm}|}
\hline
\textbf{ステートメント型} & \textbf{説明} & \textbf{例} \\
\hline
Compare & 比較演算 & x > 0, x < 0 \\
\hline
Call & 関数呼び出し & print(i), range(x) \\
\hline
Assignment & 代入 & i = value \\
\hline
Return & return文 & return x \\
\hline
Nop & 特殊操作 & FUNCTION\_START, FUNCTION\_END \\
\hline
UnsupportedStmt & サポート外 & x += 1, CONTROL\_STRUCTURE \\
\hline
\caption{ステートメントの種類}
\end{longtable}

\section{基本的な使用例}

\subsection{サンプルコード}

\begin{lstlisting}
# example.py
def example(x):
    if x > 0:
        for i in range(x):
            print(i)
    else:
        while x < 0:
            x += 1
    return x
\end{lstlisting}

\subsection{基本的な解析スクリプト}

\begin{lstlisting}
from pyjoern import parse_source
import networkx as nx

def basic_analysis(file_path):
    """基本的な解析"""

    # ファイルを解析
    functions = parse_source(file_path)

    for func_name, func_obj in functions.items():
        print(f"\n=== 関数: {func_name} ===")
        print(f"開始行: {func_obj.start_line}")
        print(f"終了行: {func_obj.end_line}")

        # CFGの基本情報
        if func_obj.cfg:
            cfg = func_obj.cfg
            print(f"CFGノード数: {len(cfg.nodes)}")
            print(f"CFGエッジ数: {len(cfg.edges)}")

            # 各ノードの詳細
            print("\nCFGノード詳細:")
            for i, node in enumerate(cfg.nodes):
                print(f"  ノード{i+1}: Block:{node.addr}")
                print(f"    エントリー: {node.is_entrypoint}")
                print(f"    エグジット: {node.is_exitpoint}")

                # ステートメント
                if hasattr(node, 'statements') and node.statements:
                    for j, stmt in enumerate(node.statements):
                        print(f"      文{j+1}: [{type(stmt).__name__}] {stmt}")

# 実行
if __name__ == "__main__":
    basic_analysis("example.py")
\end{lstlisting}

\subsection{特定の情報を取得する例}

\begin{lstlisting}
def get_basic_metrics(file_path):
    """基本メトリクスを取得"""

    functions = parse_source(file_path)

    for func_name, func_obj in functions.items():
        if func_obj.cfg:
            cfg = func_obj.cfg

            # 基本メトリクス
            nodes = len(cfg.nodes)
            edges = len(cfg.edges)

            # 制御構造をカウント
            compare_count = 0
            call_count = 0
            assignment_count = 0

            for node in cfg.nodes:
                if hasattr(node, 'statements'):
                    for stmt in node.statements:
                        stmt_type = type(stmt).__name__

                        if stmt_type == 'Compare':
                            compare_count += 1
                        elif stmt_type == 'Call':
                            call_count += 1
                        elif stmt_type == 'Assignment':
                            assignment_count += 1

            print(f"関数: {func_name}")
            print(f"  ノード数: {nodes}")
            print(f"  エッジ数: {edges}")
            print(f"  比較文: {compare_count}")
            print(f"  関数呼び出し: {call_count}")
            print(f"  代入文: {assignment_count}")
\end{lstlisting}

\subsection{高速CFGでの解析例}

\begin{lstlisting}
def fast_analysis(file_path):
    """高速CFG解析"""

    cfgs = fast_cfgs_from_source(file_path)

    for cfg_name, cfg in cfgs.items():
        print(f"\nCFG: {cfg_name}")
        print(f"ノード数: {len(cfg.nodes)}")
        print(f"エッジ数: {len(cfg.edges)}")

        # ノードの簡単な情報
        for i, node in enumerate(cfg.nodes):
            print(f"  ノード{i+1}: {repr(node)}")
\end{lstlisting}

\section{よくある質問}

\subsection{Q1: parse\_source と fast\_cfgs\_from\_source のどちらを使うべき?}

\textbf{A1:}
\begin{itemize}
    \item \textbf{詳細分析が必要} → parse\_source()
    \item \textbf{大量ファイル処理} → fast\_cfgs\_from\_source()
    \item \textbf{学習・実験目的} → parse\_source()
\end{itemize}

\subsection{Q2: エラーが発生した場合の対処法は?}

\textbf{A2:}
\begin{lstlisting}
try:
    functions = parse_source(file_path)
except Exception as e:
    print(f"解析エラー: {e}")
    # フォールバック処理
\end{lstlisting}

\subsection{Q3: 特定のステートメント型だけを取得したい}

\textbf{A3:}
\begin{lstlisting}
def get_compare_statements(cfg):
    """Compare文のみを取得"""
    compares = []

    for node in cfg.nodes:
        if hasattr(node, 'statements'):
            for stmt in node.statements:
                if type(stmt).__name__ == 'Compare':
                    compares.append(str(stmt))

    return compares
\end{lstlisting}

\subsection{Q4: ノードの属性がわからない}

\textbf{A4:}
\begin{lstlisting}
# ノードの全属性を確認
for node in cfg.nodes:
    print(f"利用可能な属性: {dir(node)}")
    print(f"__dict__: {node.__dict__}")
\end{lstlisting}

\subsection{Q5: NetworkXエラーが出る}

\textbf{A5:}
\begin{lstlisting}[language=bash]
pip install networkx
\end{lstlisting}

\section{まとめ}

PyJoernの基本的な使い方:

\begin{enumerate}
    \item \textbf{インストール}: pip install pyjoern networkx
    \item \textbf{基本API}: parse\_source() または fast\_cfgs\_from\_source()
    \item \textbf{データ取得}: 関数情報、CFG、AST、ノード詳細
    \item \textbf{解析処理}: ステートメント型に基づく分析
\end{enumerate}

\textbf{次のステップ}: より高度な分析(循環複雑度、変数分析など)は応用編で学習

\section{参考文献}

\begin{itemize}
    \item PyJoern GitHub: \url{https://github.com/octopus-platform/pyjoern}
    \item Joern公式ドキュメント: \url{https://joern.io/}
    \item NetworkX Documentation: \url{https://networkx.org/}
\end{itemize}


\section{はじめに}

静的解析によるPythonコードの特徴量抽出には、主に2つのアプローチがあります:
\begin{itemize}
    \item \textbf{PyJoern}: PythonライブラリとしてのJoern
    \item \textbf{Joern-CLI}: コマンドライン版のJoern
\end{itemize}

本ガイドでは、両者の違いと特徴量抽出に最適な選択を解説します。

\section{PyJoern vs Joern-CLI 詳細比較}

\subsection{基本比較表}

\begin{table}[h!]
\centering
\begin{tabular}{|l|l|l|}
\hline
\textbf{項目} & \textbf{PyJoern} & \textbf{Joern-CLI} \\
\hline
\textbf{言語} & Python & Scala/Java \\
\hline
\textbf{インターface} & ライブラリ & コマンドライン \\
\hline
\textbf{インストール} & pip install & バイナリダウンロード \\
\hline
\textbf{学習コスト} & 低い(Python慣れ) & 高い(専用言語) \\
\hline
\textbf{機能の完全性} & 部分的 & 完全 \\
\hline
\textbf{カスタマイズ性} & 高い & 高い \\
\hline
\textbf{大量処理} & 得意 & 得意 \\
\hline
\textbf{メモリ効率} & 中程度 & 高い \\
\hline
\textbf{処理速度} & 中程度 & 高速 \\
\hline
\end{tabular}
\caption{PyJoern vs Joern-CLI 基本比較}
\end{table}

\subsection{機能比較詳細}

\begin{table}[h!]
\centering
\begin{tabular}{|l|c|c|c|}
\hline
\textbf{機能} & \textbf{PyJoern} & \textbf{Joern-CLI} & \textbf{備考} \\
\hline
\textbf{CFG生成} & ✓ & ✓ & 両方対応 \\
\hline
\textbf{AST生成} & ✓ & ✓ & 両方対応 \\
\hline
\textbf{DDG生成} & ✓ & ✓ & データ依存グラフ \\
\hline
\textbf{PDG生成} & × & ✓ & PyJoernでは未実装 \\
\hline
\textbf{CPG統合} & 部分的 & ✓ & Joern-CLIが完全 \\
\hline
\textbf{クエリ言語} & × & ✓ & Joern専用DSL \\
\hline
\textbf{バッチ処理} & ✓ & ✓ & 両方得意 \\
\hline
\textbf{可視化} & 手動 & 組み込み & \\
\hline
\end{tabular}
\caption{機能比較詳細}
\end{table}

\section{PyJoernの特徴と制限}

\subsection{PyJoernで取得可能なグラフ}

\begin{lstlisting}
from pyjoern import parse_source

functions = parse_source("example.py")
func_obj = functions["example"]

# 取得可能なグラフ
print(f"CFG: {hasattr(func_obj, 'cfg')}")    # ✓ 利用可能
print(f"AST: {hasattr(func_obj, 'ast')}")    # ✓ 利用可能
print(f"DDG: {hasattr(func_obj, 'ddg')}")    # ✓ 利用可能
print(f"PDG: {hasattr(func_obj, 'pdg')}")    # × 未実装
\end{lstlisting}

\subsection{PyJoernで省略されている機能}

\begin{itemize}
    \item \textbf{PDG(Program Dependence Graph)}: 制御依存とデータ依存の統合グラフ
    \item \textbf{完全なCPG}: Code Property Graphの統合ビュー
    \item \textbf{Joernクエリ言語}: 専用DSLによる高度クエリ
    \item \textbf{セマンティック解析}: より深いコード理解
    \item \textbf{組み込み可視化}: グラフの自動可視化機能
\end{itemize}

\subsection{PyJoernの利点}

\begin{itemize}
    \item \textbf{Python統合}: 機械学習パイプラインに組み込みやすい
    \item \textbf{簡単なインストール}: pip一発でインストール完了
    \item \textbf{カスタマイズ性}: Pythonコードで自由に拡張
    \item \textbf{NetworkX互換}: 豊富なグラフアルゴリズムを活用
\end{itemize}

特徴量抽出が主目的であれば、\textbf{PyJoernが最適解}と言えるでしょう。

\end{document}
